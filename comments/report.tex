\documentclass{article}

\title{Rapport d'echange : Ecole Normale Superieure}
\author{Christophe Saad}
\date{\today}


\begin{document}


\maketitle

The implentation of the sequencial part is manly taken from the paper  A poor man’s concurrency monad. We consider the process as a monad transformer which takes an action (which contains the actual computation) and link it a continuation. Its type is $$\texttt{type 'a process = ('a -> action) -> action}$$
The type action is $$\texttt{type action  = Atom of (unit -> action) | Fork of action * action | Stop}$$
We use \texttt{Fork} to instanciate new processes and \texttt{Atom} to represent a computation.\\
In the \texttt{run} function, we recreate a pipeline containing the first \texttt{Fork} action to execute. Each time a \texttt{Fork} is executed, we push the two actions in the pipeline. When an \texttt{Atom} is read, we execute its computation and store his continuation back in the pipeline. This procedures ends when all the continuations are \texttt{Stop}.














\end{document}
